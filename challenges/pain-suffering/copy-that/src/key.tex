
% pro tip: don't use $ or $$

\documentclass{article}

\usepackage[utf8]{inputenc}

\usepackage{amsfonts, amsmath, amssymb, amsthm}

\usepackage{graphicx}

\usepackage[backend=biber, style=alphabetic, sorting=nty]{biblatex}

\setlength{\parindent}{0pt}
\setlength{\parskip}{5pt}
\emergencystretch=1em

\graphicspath{{./}}

\addbibresource{source.bib}

\title{Local Inversion Theorem}
\author{Charger Blue}
\date{\includegraphics[scale=0.2]{charger_blue.jpg}} % scale is 0.2

\begin{document}

\maketitle
\begin{abstract}
    The Local Inversion Theorem gives sufficient conditions for a function to be a local diffeomorphism of a point in its domain.  The conditions for the theorem are that the function is of class \(C^1\) and its derivative at the point is an isomorphism.
\end{abstract}

\section{Introduction}

The Local Inversion Theorem is stated as follows:

\textbf{Theorem (}\textit{Local Inversion}\textbf{):} Let \(E,F\) be Banach Spaces and \(U,V\) be non-empty open subsets of \(E\) and \(F\) respectfully.  If \(f:U\to V\) is of class \(C^1\) and if \(a\in U\) such that \(df(a)\) is an isomorphism, then there exists an open neighborhood, \(U_a\), of \(a\) in \(U\) and an open neighborhood, \(V_b\), of \(b=f(a)\) in \(V\) such that \(f|_{U_a}\) is a diffeomorphism from \(U_a\) onto \(V_b\).

\section{Preliminaries}

The proof of the Local Inversion Theorem requires some machinery, mainly a corollary to the Open Mapping Theorem and the Banach-Picard Theorem.  

\textbf{Theorem (}\textit{Open Mapping}\textbf{):} Let \(E,F\) be Banach Spaces.  If \(L:E\to F\) is a surjective continuous linear mapping, then \(L\) is an open mapping. So if \(U\subseteq E\) is open, then \(L(U)\) is open.\\
\textbf{Note:} A proof is provided in Chapter 8 Appendix 1 of \cite{coleman}.

\textbf{Corollary:} Let \(E,F\) be Banach Spaces.  If \(L:E\to F\) is a bijective continuous linear mapping, then the inverse of \(L\) is continuous.\\
\textbf{Note:} A proof is provided in Chapter 8 Appendix 1 of \cite{coleman}.

\textbf{Theorem (}\textit{Banach-Picard}\textbf{):} Let \(C\) be a non-empty closed subset of a Banach Space \(E\).  If \(h:C\to C\) is a strict contraction, then there exists a unique \(x\in C\) such that \(h(x)=x\).

\section{Proof}

Before giving the proof, a general outline goes as follows:
\begin{enumerate}
    \item Show \((df(a))^{-1}\) is continuous.
    \item Show \(f\) is invertible in a neighborhood of \(a\).
    \item Show \(f^{-1}\) is \(C^1\).
    \item Show \(f\) is a diffeomorphism when restricted to the neighborhood of \(a\).
\end{enumerate}

Let \(E,F\) be Banach Spaces.  Let \(U\subseteq E\) and \(V\subseteq F\) be non-empty open subsets.  Finally, let \(a\in U\).

Suppose \(f:U\to V\) is of class \(C^1\) and \(df(a)\) is an isomorphism.

\textbf{Step 1}

As \(f\) is of class \(C^1\), \(df(a)\) is continuous.  Also, because \(df(a)\)) is an isomorphism, \(df(a)\) is bijective.  This means that \(df(a)\) is a bijective continuous linear mapping.  Therefore by the corollary to the Open Mapping Theorem, \((df(a))^{-1}\) is continuous.

\textbf{Step 2}

As stated before, \(df(a)\) is continuous.  This means \(\exists r>0\) such that \(\forall x\in\bar{B}_E(a;r)\),
\[\Vert df(a)-df(x)\Vert_{\mathcal{L}(E,F)} < \frac{1}{2\Vert(df(a))^{-1}\Vert}_{\mathcal{L}(F,E)}\]

Let \(y\in B_F(f(a);\frac{r}{2\Vert(df(a))^{-1}\Vert_{\mathcal{L}(F,E)}})\).  

Define \(\phi_y:\bar{B}_E(a;r)\to E\) by
\[\phi_y(x)=x+(df(a))^{-1}(y-f(x))\]
The derivative of this mapping is
\begin{align*}
    d\phi_y(x)&=id_E-(df(a))^{-1}(df(x))\\
    &=(df(a))^{-1}(df(a)-df(x))
\end{align*}
So, for any \(x\in\bar{B}_E(a;r)\),
\begin{align*}
    \Vert d\phi_y(x)\Vert_{\mathcal{L}(E,E)}&=\Vert(df(a))^{-1}(df(a)-df(x))\Vert_{\mathcal{L}(E,E)}\\
    &\leq\Vert(df(a))^{-1}\Vert_{\mathcal{L}(F,E)}\Vert(df(a)-df(x))\Vert_{\mathcal{L}(E,F)}\\
    &<\Vert(df(a))^{-1}\Vert_{\mathcal{L}(F,E)}\frac{1}{2\Vert(df(a))^{-1}\Vert_{\mathcal{L}(F,E)}}\\
    &=\frac{1}{2}
\end{align*}
Then, by using the Mean Value Inequality, for any \(v,w\in\bar{B}_E(a;r)\)
\[\Vert\phi_y(v)-\phi_y(w)\Vert_E\leq\frac{1}{2}\Vert v-w\Vert_E\]
This means \(\phi_y\) is a strict contraction.  The codomain of \(\phi_y\) can also be redefined to \(\bar{B}_E(a;r)\) because
\begin{align*}
    \Vert\phi_y(x)-a\Vert_E&\leq\Vert\phi_y(x)-\phi_y(a)\Vert_E +\Vert\phi_y(a)-a\Vert_E\\
    &\leq\frac{1}{2}\Vert x-a\Vert_E +\Vert(df(a))^{-1}\Vert_{\mathcal{L}(F,E)}\Vert y-f(a)\Vert_F\\
    &<\frac{r}{2}+\Vert(df(a))^{-1}\Vert_{\mathcal{L}(F,E)}\frac{r}{2\Vert(df(a))^{-1}\Vert_{\mathcal{L}(F,E)}}\\
    &=r
\end{align*}
This also means \(\phi_y(\bar{B}_E(a;r))\subseteq B_E(a;r)\).

Using the Banach-Picard Theorem, there exists a unique \(x\in\bar{B}_E(a;r)\) such that \(\phi_y(x)=x\).  This means,
\begin{align*}
    &\phi_y(x)=x\\
    &\Rightarrow x+(df(a))^{-1}(y-f(x))=x\\
    &\Rightarrow (df(a))^{-1}(y-f(x))=0\\
    &\Rightarrow y=f(x)
\end{align*}
A note to make is that \(x\in B_E(a;r)\) because \(\phi_y(\bar{B}_E(a;r))\subseteq B_E(a;r)\).

Let \(U_a=f^{-1}(B_F(f(a);\frac{r}{2\Vert(df(a))^{-1}\Vert_{\mathcal{L}(F,E)}}))\cap B_E(a;r)\).  \(U_a\) is an open neighborhood of \(a\).  Given the above, \(f\) is invertible on \(U_a\).  Also, let \(V_b=B_F(f(a);\frac{r}{2\Vert(df(a))^{-1}\Vert_{\mathcal{L}(F,E)}})\).

So, \(f|_{U_a}\) is a continuous bijection from \(U_a\) onto \(V_b\).

\textbf{Step 3}

A reasonable guess for \(df^{-1}(y)\) for some \(y\in V_b\) would be \((df(f^{-1}(y)))^{-1}\).

To show this, let \(y\in V_b\) and \(k\in F\).  Then, there will exist a unique \(x\in U_a\) such that \(f(x)=y\) and \(h\in E\) such that \(f(x+h)=f(x)+k\).

Looking at the definition of the derivative, if
\[\lim_{k\to 0}\frac{\Vert f^{-1}(y+k)-f^{-1}(y)-(df(f^{-1}(y)))^{-1}k\Vert_E}{\Vert k\Vert_F}=0\]
Then, \(f^{-1}\) is derivable at \(y\).

One property of the \(\phi_y\) defined above, is,
\begin{align*}
    &\Vert\phi_y(x+h)-\phi_y(x)\Vert_E\leq\frac{1}{2}\Vert x+h-x\Vert_E\\
    &\Rightarrow\Vert x+h+(df(a))^{-1}(y-f(x+h))-x-(df(a))^{-1}(y-f(x))\Vert_E\leq\frac{1}{2}\Vert h\Vert_E\\
    &\Rightarrow\Vert h-(df(a))^{-1}(f(x+h)-f(x))\Vert_E\leq\frac{1}{2}\Vert h\Vert_E\\
    &\Rightarrow\Vert h\Vert_E-\Vert(df(a))^{-1}(f(x+h)-f(x))\Vert_E\leq\frac{1}{2}\Vert h\Vert_E\\
    &\Rightarrow\Vert h\Vert_E\leq 2\Vert(df(a))^{-1}\Vert_{\mathcal{L}(F,E)}\Vert f(x+h)-f(x)\Vert_F\\
    &\Rightarrow\Vert h\Vert_E\leq 2\Vert(df(a))^{-1}\Vert_{\mathcal{L}(F,E)}\Vert k\Vert_F
\end{align*}
So as \(k\to 0\), \(h\to 0\).

This means,
\begin{align*}
    &\frac{\Vert f^{-1}(y+k)-f^{-1}(y)-(df(f^{-1}(y)))^{-1}k\Vert_E}{\Vert k\Vert_F}\\
    &=\frac{\Vert f^{-1}(f(x)+f(x+h)-f(x))-f^{-1}(f(x))-(df(x))^{-1}k\Vert_E}{\Vert k\Vert_F}\\
    &=\frac{\Vert x+h-x-(df(x))^{-1}k\Vert_E}{\Vert k\Vert_F}\\
    &=\frac{\Vert h-(df(x))^{-1}k\Vert_E}{\Vert k\Vert_F}\\
    &=\frac{\Vert (df(x))^{-1}((df(x))h-f(x+h)+f(x))\Vert_E}{\Vert k\Vert_F}\\
    &=\frac{\Vert (df(x))^{-1}(f(x+h)-f(x)-df(x)h)\Vert_E}{\Vert k\Vert_F}\\
    &\leq\Vert (df(x))^{-1}\Vert_{\mathcal{L}(F,E)}\frac{\Vert(f(x+h)-f(x)-df(x)h)\Vert_F}{\Vert k\Vert_F}\\
    &=\Vert (df(x))^{-1}\Vert_{\mathcal{L}(F,E)}\frac{\Vert h\Vert_E}{\Vert k\Vert_K}\frac{\Vert(f(x+h)-f(x)-df(x)h)\Vert_F}{\Vert h\Vert_E}\\
    &\leq2\Vert (df(a))^{-1}\Vert_{\mathcal{L}(F,E)}\Vert (df(x))^{-1}\Vert_{\mathcal{L}(F,E)}\frac{\Vert k\Vert_F}{\Vert k\Vert_F}\frac{\Vert (f(x+h)-f(x)-df(x)h)\Vert_F}{\Vert h\Vert_E}\\
    &=2\Vert (df(a))^{-1}\Vert_{\mathcal{L}(F,E)}\Vert (df(x))^{-1}\Vert_{\mathcal{L}(F,E)}\frac{\Vert(f(x+h)-f(x)-df(x)h)\Vert_F}{\Vert h\Vert_E}
\end{align*}
This will go to zero as \(k\to 0\) because
\[\lim_{h\to 0}\frac{\Vert(f(x+h)-f(x)-df(x)h)\Vert_F}{\Vert h\Vert_E}=0\]
and as \(k\to 0\), \(h\to 0\).

So,
\[\lim_{k\to 0}\frac{\Vert f^{-1}(y+k)-f^{-1}(y)-(df(f^{-1}(y)))^{-1}k\Vert_E}{\Vert k\Vert_F}=0\]
and \(f^{-1}\) is derivable at any \(y\in V_b\) and \(df^{-1}(y)=(df(f^{-1}(y)))^{-1}\).

Finally, because \(f^{-1}\) is continuous, \(df\) is continuous, and \(Q:\mathcal{I}(E,F)\to\mathcal{L}(F,E)\), \(Q(x)=x^{-1}\) is continuous, \(df^{-1}(y)=Q\circ df\circ f^{-1}(y)\) is continuous. In other words \(f^{-1}\) is \(C^1\).

\textbf{Step 4}

\(f|_{U_a}:U_a\to V_b\) is a \(C^1\)-diffeomorphism because \(f|_{U_a}\) is bijective, \(f|_{U_a}\) is \(C^1\), and \((f|_{U_a})^{-1}\) is \(C^1\).

Therefore, there exists an open neighborhood, \(U_a\), of \(a\) in \(U\) and an open neighborhood, \(V_b\), of \(b=f(a)\) in \(V\) such that \(f|_{U_a}\) is a diffeomorphism from \(U_a\) onto \(V_b\).

\section{Example}

Let \(f:\mathbb{R}^2\setminus\{(0,0)\}\to\mathbb{R}^2\) be defined by \(f(r,\theta)=(r\cos\theta,r\sin\theta)\).  This means the Jacobian of \(f\) is
\[J_f(r,\theta)=\begin{pmatrix}\cos\theta &-r\sin\theta\\\sin\theta &r\cos\theta\end{pmatrix}\]
So the determinant of \(J_f(r,\theta)\) is equal to \(r\) which is non-zero.  This means \(J_f(r,\theta)\) is invertible for any \((r,\theta)\in\mathbb{R}^2\setminus\{(0,0)\}\).  An invertible matrix is an isomorphism and clearly \(f\) is \(C^1\), so the Local Inversion Theorem can be applied.

Let \(a=(r,\theta)\).  Then,
\[J_f(r,\theta)=\begin{pmatrix}\cos\theta &-r\sin\theta\\\sin\theta &r\cos\theta\end{pmatrix}\]
and the determinant,
\[\det J_f(r,\theta)=r\cos^2\theta+r\sin^2\theta=r\]
Inverting the Jacobian
\[J_f(r,\theta)^{-1}=\begin{pmatrix}\cos\theta &\sin\theta\\-\frac{1}{r}\sin\theta &\frac{1}{r}\cos\theta\end{pmatrix}\]
Finally, one local inversion of the function in \(H=\{(x,y)\in\mathbb{R}^2:x>0\}\),
\[f^{-1}(x,y)=(\sqrt{x^2+y^2},\arctan(y/x))\]
Clearly this is only one way to invert the function, specifically for \(-\pi/2<\theta<\pi/2\).  If you wanted to invert the function for some \(3\pi/2<\theta<5\pi/2\), then the inverse would change.  In this case it would be
\[f^{-1}(x,y)=(\sqrt{x^2+y^2},2\pi+\arctan(y/x))\]

\section{Results}

The Local Inversion Theorem has some interesting extensions from it.

The Global Inversion Theorem is one of these extensions:

\textbf{Theorem (}\textit{Global Inversion}\textbf{):} Let \(E,F\) be Banach Spaces and \(U,V\) be non-empty open subsets of \(E\) and \(F\) respectfully.  Let \(f:U\to V\) is of class \(C^1\).  \(f:U\to f(U)\) is a diffeomorphism if and only if \(f\) is injective and \(df(a)\) is an isomorphism for any \(a\in U\).

This extension come very naturally from the Local Inversion Theorem, as it is essentially applying the Local Inversion Theorem to every point in \(U\).

A second extension is into \(C^k\) functions, where \(k>1\).  If \(f\) is of class \(C^k\), then the diffeomorphism provided by the Local Inversion Theorem is also \(C^k\).

\newpage
\printbibliography

\end{document}